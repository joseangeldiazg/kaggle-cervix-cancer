%---------------------------------------------------
% Nombre: capitulo1.tex  
% 
% Texto del capitulo 1
%---------------------------------------------------

\chapter{Introducci�n}

Esta �ltima pr�ctica est� enmarcada dentro de la asignatura \textbf{Sistemas Inteligentes para La Gesti�n en la Empresa} del Master Profesional en Ingenier�a Inform�tica de la UGR  y aborda un problema real de predicci�n multiclase en la plataforma Kaggle \cite{kaggle}. 

Este problema, es de un nivel avanzado, y a lo largo de los siguientes cap�tulos intentaremos aportar una soluci�n aceptable en la plataforma Kaggle, as� como estudiar y asentar los diferentes conceptos te�ricos vistos en la asignatura. 


\section{Problema y Dataset}
\label{dataset}

El problema en �ltima instancia es un problema de clasificaci�n multiclase real el cual deber� ser resuelto mediante t�cnicas de \textit{deeplearning}. El problema, en concreto es Intel & MobileODT Cervical Cancer Screening \cite{challenge} y trata de clasificar partiendo de im�genes del cervix de distintas pacientes, que tipo de tratamiento para el cancer es m�s efectivo. 


\section{Herramientas y objetivos}

En esta secci�n veremos una breve introducci�n a las herramientas usadas para el desarrollo de la pr�ctica as� como de los principales objetivos que se buscan conseguir con el desarrollo de la misma. 

\subsection{Hardware}ls

\begin{table}[H]
	\begin{center}
		\begin{tabular}{cc} \toprule
		Elemento & Caracter�sticas \\ \midrule
		Procesador & 2,6 GHz Intel Core i5 \\
		GPU & - \\
		Memoria Ram &  8 GB 1600 MHz DDR3 \\
		Disco duro & SATA SSD de 120 GB\\ \bottomrule
		\end{tabular}
	\end{center}
\caption{Especificaciones t�cnicas de la m�quina 1.}
\label{ord_personal1}
\end{table}

\begin{table}[H]
	\begin{center}
		\begin{tabular}{cc} \toprule
		Elemento & Caracter�sticas \\ \midrule
		Procesador & 2,6 GHz Intel Core i5 \\
		GPU & 2,6 GHz Intel Core i5 \\
		Memoria Ram &  8 GB 1600 MHz DDR3 \\
		Disco duro & SATA SSD de 120 GB\\ \bottomrule
		\end{tabular}
	\end{center}
\caption{Especificaciones t�cnicas de la m�quina 2.}
\label{ord_personal2}
\end{table}

\subsection{Software}

El software utilizado es en su pr�ctica totalidad software libre, siendo el restante software propietario cuyas licencias vienen incluidas en el sistema operativo de la m�quina usada siendo este OS X "Sierra". El software usado es:

\begin{itemize}
	\item \textbf{RStudio}\cite{rstudio}: Entorno de trabajo para R.
	\item \textbf{TeXShop}, procesador de textos basado en \textit{Latex} usado para elaborar la documentaci�n del presente proyecto.
\end{itemize}

\subsection{Objetivos}

Los objetivos de este trabajo podr�an resumirse en los siguientes:

\begin{itemize}
	\item Obtener un modelo predictivo binario fiable que dado un nuevo pasajero X con sus caracter�sticas dadas,  sea capaz de predecir si habr�a sobrevivido o no al desastre del Titanic. 
	\item Aumentar el accuracy de nuestro sistema de manera que podamos subir posiciones en la competici�n relativa en Kaggle \cite{kaggle}. 
	\item Comprender y estudiar las distintas t�cnicas de miner�a de datos vistas en la asignatura. 
\end{itemize}

\section{Organizaci�n del trabajo} 

La organizaci�n del presente documento, se centra en detallar cada uno de los pasos seguidos durante el estudio y resoluci�n del problema planteado en esta introducci�n. En primera instancia, encontramos en el cap�tulo \ref{1} un estudio explorativo y estad�stico del dataset, continuaremos con modelos simples basados en reglas y en predicciones realizadas a simple vista con el conocimiento extraido de la exploraci�n para continuar con modelos avanzados de clasificaci�n en el cap�tulo \ref{clasificacion} y finalizar con v�as futuras y una tabla resumen con los resultados del proceso experimental en el anexo \ref{anexo}.

\clearpage
%---------------------------------------------------