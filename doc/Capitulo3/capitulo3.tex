%?????????????????????????
% Nombre: capitulo3.tex  
% 
% Texto del capitulo 3
%---------------------------------------------------

\chapter{Clasificaci�n con NN}
\label{nn}

En este capitulo veremos el proceso seguido y las distintas vertientes de entrenamiento usadas a lo largo de la realizaci�n de la pr�ctica. Concretamente veremos entrenamiento \textit{from scratch} con dos tecnolog�as distintas y por �ltimo \textit{fine tunning}. 


\section{Intel Deeplearning SDK}

Como el software se ve en la pr�ctica y en virtud de probar todos y cada uno de los elementos estudiados, instalamos el software de Intel en la m�quina \ref{ord_personal1}. El primero de los problemas vino con el tiempo de instalaci�n y la cantidad de memoria necesaria por este, por un lado para desplegar los 4 contenedores docker que montan la app, como para alojar los dataset, ya que crea una copia dentro de los contenedores de los mismos. Por otro lado, una vez instalado, constatamos la imposibilidad de validar el conjunto de test de la competici�n sin realizar modificaciones en los scripts de Caffe o Tensorflow o creando clases ficticias para que clasifique con las cuales el proceso de entrenamiento ofrecer�a resultados muy malos y la interfaz gr�fica del mismo servir�a mas bien de poco. La cantidad de problemas y la perdida de tiempo con el software, hizo que migr�ramos directamente hacia otras tecnolog�as como las que veremos a continuaci�n. 

\section{From Scratch}

La estrategia seguida en una competici�n en la que el tiempo de computo era un factor claramente privativo, ha pasado por afinar el proceso de preprocesado de datos con scripts que ofrec�an resultados 'aceptables` en poco tiempo de computo. Por ello, nos hemos basado en redes neuronales muy sencillas, sobre las cuales podr�amos ir probando distintas combinaciones de preprocesado, como las vistas en el cap�tulo \ref{preprocesado} y ver cuales son las que mejor se adaptan a nuestro problema, con la premisa de si funcionan bien en una red sencilla tambi�n lo har�n en redes mas complejas. Las redes elaboradas provienen de kernels de Kaggle, concretamente la de MXNET proviene del siguiente tutorial \cite{tutorial1}  y la de TensorFlow de este otro \cite{tutorial2}, aunque si bien el c�digo ha sido modificado para adaptarlo a nuestros requisitos. 

\subsection{Mxnet}

El siguiente script fue 

\lstset{language=R, breaklines=true, basicstyle=\footnotesize}
\lstset{numbers=left, numberstyle=\tiny, stepnumber=1, numbersep=-2pt}
\begin{lstlisting}
          #-------------------------------------------------------------------------------
  # Kaggle Intel Cervix Cancer Challenge
  #
  #
  # Image loading and basic pre-processing with EBImage
  # Color images
  # Submission to Kaggle is generated
  #-------------------------------------------------------------------------------


  #-------------------------------------------------------------------------------
  #Load librarys
  #-------------------------------------------------------------------------------


  library(dplyr)
  library(EBImage)
  library(mxnet)
  library(nnet)


  #-------------------------------------------------------------------------------
  # Load images using EBImage
  #
  # This loop resize all images, we dont have to do data-augmentation 
  # because, it is done with the script data-augmentation.R. 
  #-------------------------------------------------------------------------------


  paths <- c("/Users/joseadiazg/Desktop/Temporales UGR/SIGE/all_data_resized/Type_1",
             "/Users/joseadiazg/Desktop/Temporales UGR/SIGE/all_data_resized/Type_2",
             "/Users/joseadiazg/Desktop/Temporales UGR/SIGE/all_data_resized/Type_3")


  # Uncomment this if you want to load all training + extra images


  #-------------------------------------------------------------------------------
  #paths <- c("/Users/joseadiazg/Desktop/Temporales UGR/SIGE/all_data_resized/Type_1",
  #           "/Users/joseadiazg/Desktop/Temporales UGR/SIGE/all_data_resized/Type_2",
  #           "/Users/joseadiazg/Desktop/Temporales UGR/SIGE/all_data_resized/Type_3",
  #            "/Users/joseadiazg/Desktop/Temporales UGR/SIGE/train-extra-unidas/Type_1",
  #            "/Users/joseadiazg/Desktop/Temporales UGR/SIGE/train-extra-unidas/Type_2",
  #            "/Users/joseadiazg/Desktop/Temporales UGR/SIGE/train-extra-unidas/Type_3",
  #            )
  #-------------------------------------------------------------------------------

  in_type_counter <- 1 

  for (t in  paths){
    
    patients <- dir(t)
    
    #For this simple example We rescale photos to 128*128 (= 16384)

    
    n_columnas <- 1 + 1 + 49152
    
    
    ordered_images <- data.frame(matrix(nrow = length(patients), ncol = n_columnas))
    colnames(ordered_images) <- c("paciente", "Type", paste("R", (1:16384), sep = ""), paste("G", (1:16384), sep = ""), paste("B", (1:16384), sep = ""))
    
    contador <- 1
    
    if(in_type_counter == 1){
      ordered_images$Type <- 1 #(*) ojo , asignaci�n manual segun el folder que est� procesando
    }
    
    if(in_type_counter == 2){
      ordered_images$Type <- 2 #(*) ojo , asignaci�n manual segun el folder que est� procesando
    }
    
    if(in_type_counter == 3){
      ordered_images$Type <- 3 #(*) ojo , asignaci�n manual segun el folder que est� procesando
    }
    
    #+++
    mom_inicio <- Sys.time()
    print("beginning calculation: ")
    print(mom_inicio)
    #+++
    
    for (p in patients){
      
      cat("contador:", contador, " paciente:", p, "\n")
      
      ordered_images$paciente[contador] <- p
      
      imagen_paciente <- readImage(paste(t, p, sep = "/"))   #abre imagen de cada paciente...
      
      #TODO: We have to change this resize to a new version in witch we use the same
      # proportion of tam but smaller. 
      
      imagen_paciente <- resize(imagen_paciente, w = 128, h = 128)
      
      
      ordered_images[contador,  c(3:16386)] <- imagen_paciente[, , 1]
      ordered_images[contador,  c(16387:32770)] <- imagen_paciente[, , 2]
      ordered_images[contador,  c(32771:49154)] <- imagen_paciente[, , 3]
      
      contador <- contador + 1 
      
    }
    
    
    #+++
    print("calculation time: ")
    print(Sys.time() - mom_inicio)
    #+++
    
    
    if(in_type_counter == 1){
      ordered_images_Type_1 <- ordered_images
    }
    
    if(in_type_counter == 2){
      ordered_images_Type_2 <- ordered_images
    }
    
    if(in_type_counter == 3){
      ordered_images_Type_3 <- ordered_images
    }
    
    
    in_type_counter <-  in_type_counter + 1 
  }


  ordered_images_all <- bind_rows(ordered_images_Type_1, ordered_images_Type_2, ordered_images_Type_3)
    

  #-------------------------------------------------------------------------------
  # Subset for validation
  #-------------------------------------------------------------------------------

  table(ordered_images_all$Type)


  ordered_images_all$Type <- ordered_images_all$Type - 1  


  set.seed(9)
  set_validacion <- ordered_images_all %>% 
    group_by(Type) %>% 
    sample_n(25) %>% 
    ungroup()


  #-------------------------------------------------------------------------------
  # Undersampling
  #-------------------------------------------------------------------------------


  set_train_unbalanced <- ordered_images_all[ordered_images_all$paciente %in% setdiff(ordered_images_all$paciente, set_validacion$paciente), ]

  #primera muestra train undersampled:
  set.seed(9)
  undersample_1_set_train <- set_train_unbalanced %>% 
    group_by(Type) %>% 
    #sample_n(225) %>% 
    ungroup() %>% 
    sample_frac(1) %>%  
    sample_frac(1) #WE SHUFFLE TWICE

  # dim(undersample_1_set_train)
  # #[1]   675 12290
  # dim(set_validacion)
  # #[1]    75 12288


  #_____train and  validaci�n labels___________________________
  target_undersample_1 <- undersample_1_set_train$Type
  label_set_validacion <- set_validacion$Type
  #____________________________________________________________


  #_____train set1(undersampled) y and validation set, both balanced_______
  undersample_1_set_train <-undersample_1_set_train[, c(3:49154)]
  set_validacion <-set_validacion[, c(3:49154)]
  #_____target y label de validaci�n_________________________________________

  undersample_1_set_train <- t(undersample_1_set_train)
  dim(undersample_1_set_train) <- c(128, 128, 3, 675)


  set_validacion <- t(set_validacion)
  dim(set_validacion) <- c(128, 128, 3, 75)




  #-------------------------------------------------------------------------------
  # Tunning parametters for the nn
  #-------------------------------------------------------------------------------


  n_output <- 3
  num_filters_conv2 = 14
  num.round = 10
  learning.rate = 0.1	
  momentum = 0.0056
  weight_decay = 0.0046	
  initializer = 0.0667	


  # AND SOME HELPER FUNCTIONS:
  mLogLoss.normalize = function(p, min_eta=1e-15, max_eta = 1.0){
    #min_eta
    for(ix in 1:dim(p)[2]) {
      p[,ix] = ifelse(p[,ix]<=min_eta,min_eta,p[,ix]);
      p[,ix] = ifelse(p[,ix]>=max_eta,max_eta,p[,ix]);
    }
    #normalize
    for(ix in 1:dim(p)[1]) {
      p[ix,] = p[ix,] / sum(p[ix,]);
    }
    return(p);
  }

  # helper function
  #calculates logloss
  mlogloss = function(y, p, min_eta=1e-15,max_eta = 1.0){
    class_loss = c(dim(p)[2]);
    loss = 0;
    p = mLogLoss.normalize(p,min_eta, max_eta);
    for(ix in 1:dim(y)[2]) {
      p[,ix] = ifelse(p[,ix]>1,1,p[,ix]);
      class_loss[ix] = sum(y[,ix]*log(p[,ix]));
      loss = loss + class_loss[ix];
    }
    #return loss
    return (list("loss"=-1*loss/dim(p)[1],"class_loss"=class_loss));
  }

  # mxnet specific logloss metric
  mx.metric.mlogloss <- mx.metric.custom("mlogloss", function(label, pred){
    p = t(pred);
    m = mlogloss(class.ind(label),p);
    gc();
    return(m$loss);
  })
    


  #-------------------------------------------------------------------------------
  # Train the nn
  #-------------------------------------------------------------------------------



  train.x <- undersample_1_set_train
  train.y <-target_undersample_1


  #____________
  data = mx.symbol.Variable('data')
  #FIRST CONVOLUITIONAL LAYER + POOLING
  conv1 = mx.symbol.Convolution(data=data, kernel=c(3, 3), num_filter = 3) 
  relu1 = mx.symbol.Activation(data=conv1, act_type="relu") 
  pool1 = mx.symbol.Pooling(data=relu1, pool_type="max", kernel=c(2,2), stride=c(2,2))


  #SECOND CONVOLUTIONAL LAYER + POOLING
  conv2 = mx.symbol.Convolution(data=pool1, kernel=c(3,3), num_filter = num_filters_conv2) 
  relu2 = mx.symbol.Activation(data=conv2, act_type="relu")
  pool2 = mx.symbol.Pooling(data=relu2, pool_type="max",kernel=c(2,2), stride=c(2,2)) 

  #FLATTEN THE OUTPUT
  flatten = mx.symbol.Flatten(data=pool2) 

  #FEED FULLY CONNECTED LAYER, NUMBER OF HIDDEN NODES JUST GEOMETRIC MEAN OF INPUT(14.5 * 14.5  * 13 =  2733.25) AND OUTPUT (3), sqrt(2733.25*3) =  91
  input_previo_a_filtroconv2 <- 14.5*14.5
  n_input <- input_previo_a_filtroconv2 * num_filters_conv2
  num_hidden_fc1 <- round(sqrt(n_input*n_output)) 

  fc1 = mx.symbol.FullyConnected(data=flatten, num_hidden=84) 
  relu4 = mx.symbol.Activation(data=fc1, act_type="relu") 

  #____________


  fc2 = mx.symbol.FullyConnected(data=relu4, num_hidden=3) #ESTA PARA CLASIFICACION

  mi_softmax  = mx.symbol.SoftmaxOutput(data=fc2)


  devices <- mx.cpu()

  mx.set.seed(0)

  #___
  tic <- proc.time()
  #___
  model <- mx.model.FeedForward.create( mi_softmax #for clasification
                                        , X=train.x
                                        , y=train.y
                                        , eval.data  = list("data" = set_validacion,"label" = label_set_validacion) 
                                        , ctx=devices
                                        , num.round=num.round
                                        , array.batch.size = 75 
                                        , learning.rate = learning.rate
                                        , momentum = momentum
                                        , wd=weight_decay
                                        , eval.metric = mx.metric.mlogloss 
                                        , initializer=mx.init.uniform(initializer) 
                                        #, epoch.end.callback = mx.callback.save.checkpoint("modelo_guardado_ccs") #(TO SAVE MODEL AT EVERY ITERATION)
                                        , batch.end.callback = mx.callback.log.train.metric(10)#, log)  
                                        , array.layout="columnmajor"
  ) 

  #___
  print(proc.time() - tic)
  #___


  #-------------------------------------------------------------------------------
  # Validation
  #-------------------------------------------------------------------------------

    
    preds <- predict(model, set_validacion,
                     ctx = NULL,
                     array.layout = "auto")
    
    #WE CAN INSPECT OUR LIMITED VALIDATION SET, PROBABILITIES:
    predicciones <- t(preds)
    predicciones <- cbind(label_set_validacion, predicciones)
    head(predicciones)
    
\end{lstlisting}


\subsection{TensorFlow+Keras}

\lstset{language=R, breaklines=true, basicstyle=\footnotesize}
\lstset{numbers=left, numberstyle=\tiny, stepnumber=1, numbersep=-2pt}
\begin{lstlisting}
            # This Python 3 environment comes with many helpful analytics libraries installed
    # It is defined by the kaggle/python docker image: https://github.com/kaggle/docker-python
    # For example, here's several helpful packages to load in

    import numpy as np # linear algebra
    import pandas as pd # data processing, CSV file I/O (e.g. pd.read_csv)
    import glob
    import os
    import matplotlib.pyplot as plt
    import seaborn as sns
    import cv2

    from multiprocessing import Pool, cpu_count
    from subprocess import check_output
    from subprocess import check_output
    from PIL import ImageFilter, ImageStat, Image, ImageDraw
    from sklearn.model_selection import GridSearchCV, StratifiedKFold
    from sklearn.preprocessing import LabelEncoder
    from sklearn.metrics import log_loss
    from keras.wrappers.scikit_learn import KerasClassifier
    from keras.models import Sequential
    from keras.layers.core import Dense, Dropout, Flatten
    from keras.layers.convolutional import Convolution2D, ZeroPadding2D, MaxPooling2D
    from keras import optimizers
    from keras.wrappers.scikit_learn import KerasClassifier
    from keras.models import Sequential
    from keras.layers.core import Dense, Dropout, Flatten, Activation
    from keras.layers.convolutional import Convolution2D, ZeroPadding2D, MaxPooling2D
    from keras import optimizers
    from keras.preprocessing.image import ImageDataGenerator
    from sklearn.model_selection import train_test_split
    from keras.optimizers import SGD
    from keras import backend as K
    K.set_image_dim_ordering('th')
    K.set_floatx('float32')


    def roi(pathtrain):
        for typ in types:
            for img in os.listdir(pathtrain + '/' + typ):
                image = pathtrain + '/'+typ+'/' + img
                os.chdir(pathtrain + '/'+typ+'/')
                ii=cv2.imread(image)
                #cv.imshow('image',ii[:,:,1])
                #cv.waitKey(0)
                b,g,r = cv2.split(ii)
                rgb_img = cv2.merge([r,g,b])
                rgb_img1 = pc.rgb_to_hsv(rgb_img)
                indices = np.where(rgb_img1[:,:,0]<0.7)
                rgb_img1[:,:,0][indices]=0
                rgb_img1[:,:,1][indices]=0
                rgb_img1[:,:,2][indices]=0
                rgb_img1 = pc.hsv_to_rgb(rgb_img1).astype(np.uint8)
                pp.imsave(fname = img.split('.')[0] + '_trans.jpg',arr = rgb_img1)
        return fname

    def im_multi(path):
        try:
            im_stats_im_ = Image.open(path)
            return [path, {'size': im_stats_im_.size}]
        except:
            print(path)
            return [path, {'size': [0,0]}]

    def im_stats(im_stats_df):
        im_stats_d = {}
        p = Pool(cpu_count())
        ret = p.map(im_multi, im_stats_df['path'])
        for i in range(len(ret)):
            im_stats_d[ret[i][0]] = ret[i][1]
        im_stats_df['size'] = im_stats_df['path'].map(lambda x: ' '.join(str(s) for s in im_stats_d[x]['size']))
        return im_stats_df


    def get_im_cv2(path):
        img = cv2.imread(path)
        resized = cv2.resize(img, (64, 64), cv2.INTER_LINEAR)
        return [path, resized]

    def normalize_image_features(paths):
        imf_d = {}
        p = Pool(cpu_count())
        ret = p.map(get_im_cv2, paths)
        for i in range(len(ret)):
            imf_d[ret[i][0]] = ret[i][1]
        ret = []
        fdata = [imf_d[f] for f in paths]
        fdata = np.array(fdata, dtype=np.uint8)
        fdata = fdata.transpose((0, 3, 1, 2))
        fdata = fdata.astype('float32')
        fdata = fdata / 255
        return fdata

    def create_model(opt_='adamax'):
        model = Sequential()
        model.add(Convolution2D(4, 3, 3, activation='relu', dim_ordering='th', input_shape=(3, 64, 64))) #use input_shape=(3, 64, 64)
        model.add(MaxPooling2D(pool_size=(3, 3), strides=(3, 3), dim_ordering='th'))
        model.add(Convolution2D(8, 3, 3, activation='relu', dim_ordering='th'))
        model.add(MaxPooling2D(pool_size=(3, 3), strides=(3, 3), dim_ordering='th'))
        model.add(Dropout(0.2))

        model.add(Flatten())
        model.add(Dense(12, activation='tanh'))
        model.add(Dropout(0.1))
        model.add(Dense(3, activation='softmax'))

        model.compile(optimizer=opt_, loss='sparse_categorical_crossentropy', metrics=['accuracy'])
        return model



    def main():
        os.chdir('D:\Facultad\Master\Segundo cuatrimestre\SIGE\PracticaFinal')
        test = glob.glob("pruebaTest/*.jpg")
        test = pd.DataFrame([[p[11:len(p)],p] for p in test], columns = ['image','path'])
        train= glob.glob("prueba1/**/*.png")+glob.glob("prueba2/**/*.jpg")
        train = pd.DataFrame([[p[0][8:14],p[0][15:len(p)],p] for p in train], columns = ['type','image','path'])

        train = im_stats(train)
        train = train[train['size'] != '0 0'].reset_index(drop=True) #remove bad images
        train_data = normalize_image_features(train['path'])

        np.save('train.npy', train_data, allow_pickle=True, fix_imports=True)

        print(len(train))

        le = LabelEncoder()
        train_target = le.fit_transform(train['type'].values)
        print(le.classes_) #in case not 1 to 3 order
        np.save('train_target.npy', train_target, allow_pickle=True, fix_imports=True)
        test_data = normalize_image_features(test['path'])

        np.save('test.npy', test_data, allow_pickle=True, fix_imports=True)

        test_id = test.image.values
        np.save('test_id.npy', test_id, allow_pickle=True, fix_imports=True)

        train_data = np.load('train.npy')
        train_target = np.load('train_target.npy')

        x_train,x_val_train,y_train,y_val_train = train_test_split(train_data,train_target,test_size=0.4, random_state=17)


        datagen = ImageDataGenerator(rotation_range=0.3, zoom_range=0.3)
        datagen.fit(train_data)


        model = create_model()
        print(x_train.shape)
        print(y_train.shape)
        model.fit_generator(datagen.flow(x_train,y_train, batch_size=15, shuffle=True), nb_epoch=35, samples_per_epoch=len(x_train), verbose=20, validation_data=(x_val_train, y_val_train))

        test_data = np.load('test.npy')
        test_id = np.load('test_id.npy')

        pred = model.predict_proba(test_data)
        df = pd.DataFrame(pred, columns=['Type_1','Type_2','Type_3'])
        df['image_name'] = test_id
        df.to_csv('submission0009.csv', index=False)

    if __name__ == '__main__':
        #freeze_support() # Optional under circumstances described in docs
        main()

\end{lstlisting}


\section{Fine Tunning}


\pagebreak
\clearpage
%---------------------------------------------------