%?????????????????????????
% Nombre: capitulo5.tex  
% 
% Texto del capitulo 5
%---------------------------------------------------

\chapter{Conclusiones y v�as futuras}
\label{conclusiones}
En este cap�tulo final se estudian los resultados obtenidos a lo largo del trabajo y v�as futuras para aumentar a�n m�s el accuracy. Tambi�n se complementan las conclusiones que se han ido obteniendo a lo largo del trabajo.

\section{V�as futuras}


Respecto a las v�as futuras, pese a que algunas ya han sido comentadas a lo largo de la memoria, cabe reunirlas en este �ltimo cap�tulo. 

Principalmente, destacar la necesidad de disponer de m�quinas m�s potentes a la hora de procesar esta gran cantidad de im�genes para poder disponer de todo el dataset sin necesidad de redimensionar y por consiguiente perdida de calidad. 

Respecto a las mejoras en la implementaci�n, ensembles de diversas topolog�as de redes neuronales con \textit{fine tunning} y prevenci�n del \textit{overfitting} ser�an muy probablemente la mejor opci�n para atacar el problema. Estas soluciones han sido estudiadas en la literatura \cite{ensembles} pero no hemos conseguido implementarlas ya que son complejas y a�n residen casi en su totalidad en el �mbito de investigaci�n de las mismas. 


\section{Conclusiones finales}

Como conclusi�n que a�na todas las dem�s podemos concluir que el problema era complejo y extenso. Se ped�an probar un gran n�mero de t�cnicas y adem�s conseguir buenos resultados en la competici�n, con lo cual, esto ha causado que se haya entrado por encima en algunas de las mismas con el fin de profundizar m�s en otras que a priori puedan parecer mejores. 

Respecto a las caracter�sticas del problema, muchas de las soluciones vistas en clase no ofrec�an buenos resultados llegando incluso algunas a no poder ejecutarse por falta de memoria, como MXNET que podia utilizar poco mas de 1GB de im�genes. Por tanto, un problema que necesita de poder procesar una gran cantidad de im�genes, donde las herramientas disponibles para su procesado son tan limitadas (ordenadores personales), dif�cilmente podr� obtener buenos resultados.

Por otro lado, cabe destacar la potencia de las librer�as de deep learning, para este tipo de problemas. Con muy poco c�digo y con conocimientos de programaci�n b�sicos, se pueden lograr grandes resultados, aunque llegar a utilizar estas herramientas nivel experto puede llegar a ser una ardua labor. 

Para finalizar las conclusiones, queremos profundizar un poco en lo trascendeltal del problema. Estamos trabajando con datos de pacientes m�dicos en los que una buena soluci�n de machine learning puede suponer salvar vidas, constatando por tanto la importancia que la ciencia de datos est� tomando en la totalidad de sectores  trascendiendo en algunos casos, como este mismo, de lo meramente inform�tico u econ�mico-empresarial.   

\pagebreak
\clearpage
%---------------------------------------------------